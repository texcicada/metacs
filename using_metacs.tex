\documentclass{article}
\usepackage{xparse}
\usepackage[table]{xcolor}
\pagecolor{green!5}
\usepackage{fontspec}
\newfontfamily{\fcmu}{Latin Modern Roman}%[Colour=blue,Scale=2]
\newfontfamily\ugfontb{Noto Sans Ugaritic}[Colour=brown,Scale=1.5]%for example


\input{metacs_input.tex} 
\input{ugtrans_input.tex} % for example
\newcommand{\mm}{\textsc{MetaCS}}
\newcommand{\ff}{\par\textsc{Format:~}}
\newcommand{\ee}{\par\textsc{Example:~}}
\newcommand{\hh}{\par\textsc{How it works:~}}
\newcommand{\rr}{\par\textsc{Reference:~}}
\rowcolors{1}{violet!15}{violet!5}
\newcommand{\reftable}[1]{%
\begin{center}
\begin{tabular}{l}
#1 
\end{tabular}
\end{center}

}
 
%======================================= 
\begin{document}
\section{Using MetaCS}
\mm ~provides commands for running and describing commands.

\section{Running commands}
%----------------------------
\subsection{\cs{\cc} -- run a parameterless command (switch)}
A switch is a control sequence (command) with no arguments, e.g. \cdrd{\itshape} which switches the current font to italic shape.
\ff\cs{\cc}\marg{command-name}, where \meta{command-name} is without the backslash.
\ee
\cdrq{{\cc{itshape} This is italic.}}
\hh \cs{\cc} uses \codedetokb{\cs:w \#1\textbackslash cs\_end:} to construct the control sequence expandably, add the backslash (\textbackslash) at the front, and then run it.
\par If the resulting control sequence does not exist, it is no error. \cdrq{\cc{xyz}}
\par Items passed in to the construction are expanded.
\cdrq{\newcommand\x{o}\newcommand\xxo{abc}\cc{xx\x}}
\rr 
\par\reftable{\codedetokb{\cs:w \#1\textbackslash cs\_end:}\\}


%----------------------------
\subsection{\cs{\cd} -- run a 1-argument command}
\cs{\cd} runs commands that take one argument, like \cdrd{\textbf{Some text}}.
\ff\cs{\cd}\marg{command-name}\marg{argument}, where \meta{command-name} is the control sequence name without the backslash.
\ee
\cdrq{\cc{textbf}{Some text}}
\cdrq{\cd{textbf}{Some more text}}
\hh\cs{\cd} builds on \cs{\cc}'s method of constructing a command name by adding an argument.
\par It is no compilation error if the constructed command is not defined.
\cdrq{\cc{textbff}{Some text}}
\rr 
%\par\reftable{\codedetokb{\textbackslash cs:w \#1\textbackslash cs\_end: \{ \textbackslash  tl\_use:N \textbackslash l\_my\_tl \}\\}}
%%\par\reftable{\codedetokbb{\textbackslash  cs:w \#1\textbackslash cs\_end: \{ \textbackslash  tl\_use:N \textbackslash l\_my\_tl \}}\\}
\par\reftable{\codedetokb{\tl\_set:Nn \textbackslash l\_my\_tl \{ \#2 \}} \\ \codedetokb{\cs:w \#1\textbackslash cs\_end: \{ \textbackslash  tl\_use:N \textbackslash l\_my\_tl \}}\\}

%----------------------------
\subsection{\cs{\cdr} -- print and run code}
\cs{\cdr} prints and runs whatever code is passed to it.
\ff\cs{\cdr}\marg{code}
\ee
\cdrq{\cdr{\sffamily\large\textsc{Some text}}}
\hh\cs{\cdr} uses \codedetok{\detokenize} to print its argument \#1, then leaves argument \#1 in the input stream.
\rr In effect
\par\reftable{\codedetokb{\detokenize\{\#1\}} \\ \codegeneral{\$\textbackslash mapsto\$} \\ \codegeneral{\#1} \\}


%---------------------------- cdrq
\subsection{\cs{\cdrq} -- print and run code in quotation environment}
\cs{\cdrq} does a \cs{\cdr} inside a \codegeneral{\textbackslash begin\{quotation\}} \ldots  \codegeneral{\textbackslash end\{quotation\}} environment.
\ff\cs{\cdrq}\marg{code}
\ee
\cdrq{\cdrq{\ttfamily\tiny\color{red} Example}}
\ee
\cdrq{\cdrq{\textsf{\textit{Another \colorbox{blue!20}{example}}}}}

%---------------------------- cdq
\subsection{\cs{\cdq} -- print code in quotation environment}
\cs{\cdq} prints code inside a \codegeneral{\textbackslash begin\{quotation\}} \ldots  \codegeneral{\textbackslash end\{quotation\}} environment.
\ff\cs{\cdq}\marg{code}
\ee
\cdrq{\cdq{\ttfamily\tiny\color{red} Example}}
\ee
\cdrq{\cdq{\textsf{\textit{Another \colorbox{blue!20}{example}}}}}





%---------------------------- cdrd
\cdrd[format=subsection][print formatted code]{cdrd}
%\subsection{\cs{\cdrd} -- print formatted code}
\cs{\cdrd} prints the code it is given. Variant typesetting formats, if desired, are selected via a key-value option. If no option is specified, then the plain style is used.
\ff\cs{\cdrd}\oarg{style-option}\oarg{option2}\marg{code}
\ee The heading for this subsection was produced with \cdrd[format=quote]{\cdrd[format=subsection][print formatted code]{cdrd}}
\hh Values of the \oargv{format=} option key determine what formatting is applied.

\bigskip
List of values for the \codegeneral{format=} key:

\begin{quotation}
\begin{tabular}{ll}
%\rowcolor{\mytheadcolour}
\bfseries Value & \bfseries Example \\
\hline
\meta{None} & \cdr{\cdrd{\xyz}}\\
head & \cdr{\cdrd[format=head]{\xyz}}\\
custom & \cdr{\cdrd[format=custom][\bfseries\tiny\sffamily]{\xyz}}\\
section & See outside the table. \\
subsection & Similar to section \\
quote & \parbox{4.5in}{\cdrd{\cdrd[format=quote]{xyz demo This is a quote}}: \\ See outside the table.} \\
%listing & <in development> \\
general & \cdr{\cdrd[format=general]{[format=xyz]}}\\
detok & \cdr{\cdrd[format=detok]{\xyz}}\\
\hline
\end{tabular}
\end{quotation}

%%-----------------: general
\subsubsection{\cs{\cdrd}: \cdrd[format=general]{[format=general]} key}
This key runs the \cs{\codegeneral} macro, which prints what it is given in \codedetok{\ttfamily} format.
\ee
\cdrq{\cdrd[format=general]{sample plain text}}

%%-----------------: detok
\subsubsection{\cs{\cdrd}: \cdrd[format=general]{[format=detok]} key}
This key runs the \cs{\codedetok} macro, which detokenizes what it is given and then prints it in \codedetok{\ttfamily} format.
\ee
\cdrq{\cdrd[format=detok]{\bfseries sample bold text}}

%%-----------------: head
\subsubsection{\cs{\cdrd}: \cdrd[format=general]{[format=head]} key}
This key runs the \cs{\cdrdhead} macro, which formats the command something like a heading.
\ee
\cdrq{\cdrd[format=head]{\xyz}}

%%-----------------: custom
\subsubsection{\cs{\cdrd}: \cdrd[format=general]{[format=custom]} key}
This key runs the \cs{\cdrdcustom} macro, which formats the command according to what was given to \oarg{option2} as switches.
\ee
\cdrq{\cdrd[format=custom][\bfseries\tiny\sffamily]{\xyz}}


%%-----------------: section
\subsubsection{\cs{\cdrd}: \cdrd[format=general]{[format=section]} key}\label{sec:secformat}
This key runs the \cs{\cdrdsection} macro, which does \codegeneral{\textbackslash section\{\#3 -- \#2\} \textbackslash label \{ sec:\#3\}}  in terms of the parameters received from the main function.
\rr\reftable{
  \textbackslash group\_begin: \\
  \textbackslash keys\_set:nn \{ codef \} \{ \#1 \} \\
  \textbackslash tl\_use:N \textbackslash l\_codeformat\_tl [ \#2 ] \{ \#3 \} \\
  \textbackslash group\_end: \\
}
\par\codegeneral{codef} receives the format option as \#1 and assigns the corresponding macro to \codegeneral{\textbackslash l\_codeformat\_tl}.
\par The \codegeneral{codef} key set is defined as follows:



\begin{latexcode}
\tl_new:N  \l_codeformat_tl
\tl_set:Nn \l_codeformat_tl { \cdrdplain }

\keys_define:nn { codef }
 {
  format .choice:,
  format / head   .code:n = \tl_set:Nn \l_codeformat_tl  { \cdrdhead },
  format / custom   .code:n = \tl_set:Nn \l_codeformat_tl  { \cdrdcustom }, %user-code will be second option
  format / section   .code:n = \tl_set:Nn \l_codeformat_tl  { \cdrdsection }, 
  format / subsection   .code:n = \tl_set:Nn \l_codeformat_tl  { \cdrdsubsection }, 
  format / quote   .code:n = \tl_set:Nn \l_codeformat_tl  { \cdrdquote }, 
  format / listing   .code:n = \tl_set:Nn \l_codeformat_tl  { \cdrdlisting }, 
  format / general   .code:n = \tl_set:Nn \l_codeformat_tl  { \codegeneral }, 
  format / detok   .code:n = \tl_set:Nn \l_codeformat_tl  { \codedetok }, 
   }
\end{latexcode}


%%-----------------: subsection
\subsubsection{\cs{\cdrd}: \cdrd[format=general]{[format=subsection]} key}
This key runs the \cs{\cdrdsubsection} macro, which does the same as \cs{\cdrdsection} but for subsections. See \S\ref{sec:secformat}.

%%-----------------: quote
\subsubsection{\cs{\cdrd}: \cdrd[format=general]{[format=quote]} key}
This key runs the \cs{\cdrdquote} macro, which runs the \cs{\cdrdplain} macro inside a \codegeneral{\textbackslash begin\{quotation\}} \ldots  \codegeneral{\textbackslash end\{quotation\}} environment.
\ee
\cdrq{\cdrd[format=quote]{\xyz inside quote environment}}

%%-----------------: plain
\subsubsection{\cs{\cdrd}: no format key}
If no format key is specified, the \cs{\cdrdplain} macro runs, which just does a \codedetok{\detokenize} formatted as  ttfamily large blue text.
\ee
\cdrq{\cdrd{\xyz plain formatting}}




%---------------------------- cdpr
\subsection{\cs{\cdpr} -- print and run command + formatted argument}
\cs{\cdpr} prints, then runs any code given as \#1 before the argument (given as \#3) of the command (given as \#2). The pre-argument insertion can be arbitrary code, but is intended to be formatting code, e.g., for emphasis, or for font commands for a different script to make the argument visible like it would be in a code editor window.
\ff\cs{\cdrq}\oarg{format-code}\marg{command-name}\marg{argument}
\ee Argument is red bold
\cdrq{\cdpr[\bfseries\color{red}]{textit}{fgh mno}}
\ee Argument is visible Ugaritic glyphs%\sugtransrev{𐎄𐎍𐎚}\cdr{\sugtransrev{𐎄𐎍𐎚}}
\cdrq{\cdpr[\ugfontb]{sugtransrev}{𐎄𐎍𐎚}}

%---------------------------- meta - emphasis
\section{Control Sequence Meta Commands}\label{sec:meta}
The parts of a control sequence can be referred to in text using metacommands.

\subsection{Meta Commands for emphasis}\label{sec:metaemph}
The following table lists the metacommands that can be used for emphasis.

\bigskip
\begin{tabular}{lll}
%\rowcolor{\mytheadcolour}
\bfseries Meta command & \bfseries Example & \bfseries Result \\
\hline
\cs{\cs}\marg{macro-name} & \cdrd{\cs{\test}} & \cs{\test} \\
\cs{\marg}\marg{mandatory argument} & \cdrd{\marg{test}} & \marg{test} \\
\cs{\margv}\marg{mandatory argument value} & \cdrd{\margv{test}} & \margv{test} \\
\cs{\oarg}\marg{optional argument} & \cdrd{\oarg{test}} & \oarg{test} \\
\cs{\oargv}\marg{optional argument value} & \cdrd{\oargv{test}} & \oargv{test} \\
\cs{\meta}\marg{meta value} & \cdrd{\meta{test}} & \meta{test} \\
\hline
\end{tabular}



\subsection{Unemphasized Meta Commands}\label{sec:metaunemph}
The following table lists the metacommands that can be used for unemphasized text.


\bigskip
\begin{tabular}{lll}
%\rowcolor{\mytheadcolour}
\bfseries Unemphasized Meta command & \bfseries Example & \bfseries Result \\
\hline
\cs{\css}\marg{cs-name} &\cdrd{\css{test}}& \css{test} \\
\cs{\margcss}\marg{mandatory argument} &\cdrd{\margcss{test}}& \margcss{test} \\
\cs{\margvcss}\marg{mandatory argument value} &\cdrd{\margvcss{test}}& \margvcss{test} \\
\cs{\oargcss}\marg{optional argument} &\cdrd{\oargcss{test}}& \oargcss{test} \\
\cs{\oargvcss}\marg{optional argument value} &\cdrd{\oargvcss{test}}& \oargvcss{test} \\
\cs{\meta}\marg{meta value} &\cdrd{\meta{test}}& \meta{test} \\
\hline
\end{tabular}

\section{Examples}
The \cs{\textit}\margvcss{} command is used on a run of inline text to produce italics.
\ee
\cdrq{some text \textit{(a) some italic text} some more text.}

The \cs{\itshape} switch switches the font over to italics. Its scope is restricted by the use of braces to form a group, or reversed by the \cs{\upshape} switch.
\ee braces
\cdrq{some text {\itshape (b) some italic text} some more text.}
\ee \css{upshape}
\cdrq{some text \itshape (c) some italic text \upshape some more text.}

The \cs{\cc}\margvcss{} command runs a switch.
\ee
\cdrq{{ \cc{itshape}\cc{large} 123 }}

The \cs{\cd}\margvcss{} command runs a command plus argument.
\ee
\cdrq{\cd{textit}{Sample}}

The \cs{\cdrd}\margvcss{} command prints its argument, and using its option-key settings, \cs{\cdrd}\oargvcss{format=??}\margvcss{}, can print in various formats.
\ee plain style
\cdrq{\cdrd{\textit{Sample}}}
\ee heading style
\cdrq{\cdrd[format=head]{\textit}}
\ee code style
\cdrq{\cdrd[format=general]{define function x; print(x); }}

And so on.

A code listing is done with the \cdrd[format=detok]{\begin{latexcode}} \ldots \cdrd[format=detok]{\end{latexcode}} envronment.
\ee
\begin{latexcode}

\ee plain style
\cdrq{\cdrd{\textit{Sample}}}
\ee heading style
\cdrq{\cdrd[format=head]{\textit}}
\ee code style
\cdrq{\cdrd[format=general]{define function x; print(x); }}

\end{latexcode}

\end{document}
	