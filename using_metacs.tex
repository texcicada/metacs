\documentclass{article}
\usepackage{xparse}
\usepackage[table]{xcolor}
\pagecolor{green!5}
\usepackage{fontspec}
\newfontfamily{\fcmu}{Latin Modern Roman}%[Colour=blue,Scale=2]

\input{metacs_input.tex} 
\newcommand{\mm}{\textsc{MetaCS}}
\newcommand{\ff}{\par\textsc{Format:~}}
\newcommand{\ee}{\par\textsc{Example:~}}
\newcommand{\hh}{\par\textsc{How it works:~}}
\newcommand{\rr}{\par\textsc{Reference:~}}
\rowcolors{1}{violet!15}{violet!5}
\newcommand{\reftable}[1]{%
\begin{center}
\begin{tabular}{l}
#1 
\end{tabular}
\end{center}

}
 
%======================================= 
\begin{document}
\section{Using MetaCS}
\mm ~provides commands for running and describing commands.

\section{Running commands}
%----------------------------
\subsection{\cs{\cc} -- run a parameterless command (switch)}
A switch is a control sequence (command) with no arguments, e.g. \cdrd{\itshape} which switches the current font to italic shape.
\ff\cs{\cc}\marg{command-name}, where \meta{command-name} is without the backslash.
\ee
\cdrq{{\cc{itshape} This is italic.}}
\hh \cs{\cc} uses \codedetokb{\cs:w \#1\textbackslash cs\_end:} to construct the control sequence expandably, add the backslash (\textbackslash) at the front, and then run it.
\par If the resulting control sequence does not exist, it is no error. \cdrq{\cc{xyz}}
\par Items passed in to the construction are expanded.
\cdrq{\newcommand\x{o}\newcommand\xxo{abc}\cc{xx\x}}
\rr 
\par\reftable{\codedetokb{\cs:w \#1\textbackslash cs\_end:}\\}


%----------------------------
\subsection{\cs{\cd} -- run a 1-argument command}
\cs{\cd} runs commands that take one argument, like \cdrd{\textbf{Some text}}.
\ff\cs{\cd}\marg{command-name}\marg{argument}, where \meta{command-name} is the control sequence name without the backslash.
\ee
\cdrq{\cc{textbf}{Some text}}
\cdrq{\cd{textbf}{Some more text}}
\hh\cs{\cd} builds on \cs{\cc}'s method of constructing a command name by adding an argument.
\par It is no compilation error if the constructed command is not defined.
\cdrq{\cc{textbff}{Some text}}
\rr 
%\par\reftable{\codedetokb{\textbackslash cs:w \#1\textbackslash cs\_end: \{ \textbackslash  tl\_use:N \textbackslash l\_my\_tl \}\\}}
%%\par\reftable{\codedetokbb{\textbackslash  cs:w \#1\textbackslash cs\_end: \{ \textbackslash  tl\_use:N \textbackslash l\_my\_tl \}}\\}
\par\reftable{\codedetokb{\tl\_set:Nn \textbackslash l\_my\_tl \{ \#2 \}} \\ \codedetokb{\cs:w \#1\textbackslash cs\_end: \{ \textbackslash  tl\_use:N \textbackslash l\_my\_tl \}}\\}

%----------------------------
\subsection{\cs{\cdr} -- print and run code}
\cs{\cdr} prints and runs whatever code is passed to it.
\ff\cs{\cdr}\marg{code}
\ee
\cdrq{\cdr{\sffamily\large\textsc{Some text}}}
\hh\cs{\cdr} uses \codedetok{\detokenize} to print its argument \#1, then leaves argument \#1 in the input stream.
\rr In effect
\par\reftable{\codedetokb{\detokenize\{\#1\}} \\ \codegeneral{\$\textbackslash mapsto\$} \\ \codegeneral{\#1} \\}


%----------------------------
\subsection{\cs{\cdrq} -- print and run code in quotation environment}
\cs{\cdrq} does a \cs{\cdr} inside a \codegeneral{\textbackslash begin\{quotation\}} \ldots  \codegeneral{\textbackslash end\{quotation\}} environment.
\ff\cs{\cdrq}\marg{code}
\ee
\cdrq{\cdrq{\ttfamily\tiny\color{red} Example}}









\end{document}
	